\section{Conclusion}
This thesis embarked on an innovative journey to apply Sequential Encounter Encoding for optimizing CNF encoding in the realm of itemset mining, specifically targeting 'At Least K' problems. Drawing inspiration and foundational methodologies from Carsten Sinz's seminal work, this exploration not only validated the adaptability of Sequential Encounter Encoding but also illuminated its efficacy in addressing intricate data mining challenges. Through a methodical deconstruction of the encoding process and the adept adaptation to CNF optimization, the research successfully demonstrated how at least
$\lambda$ transactions could be guaranteed as true, marking a significant stride in the application of logical encoding strategies within itemset mining.

The intricate process of constructing registers and formulating precise constraints was pivotal in modeling the 'At Least K' requirement accurately. This structured approach underscored the necessity of a deep, nuanced comprehension of encoding techniques to effectively tackle such complex problems. The findings of this thesis affirm the robustness of Sequential Encounter Encoding in optimizing CNF encodings, showcasing its capacity to efficiently manage 'At Least K' scenarios and highlighting its potential as a powerful tool in the data mining toolkit.
