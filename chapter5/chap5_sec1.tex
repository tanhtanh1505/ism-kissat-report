\section{Conclusion}

In conclusion, this study has provided a comprehensive exploration of itemset mining tasks and introduced innovative techniques for optimizing the encoding process. We began by elucidating the fundamentals of itemset mining, highlighting its significance in various domains such as retail, healthcare, and bioinformatics. The encoding of itemset mining problems into SAT instances was meticulously examined, showcasing the sequential encounter encoding approach as a promising alternative to traditional methods. By leveraging the sequential nature of transactional data, sequential encounter encoding offers a more efficient and scalable solution for itemset mining tasks.

Furthermore, we delved into the application of the standard method $C_{n-k+1}$
for encoding itemset mining constraints, shedding light on its limitations and challenges, particularly in handling large datasets. Through a comparative analysis, we demonstrated the advantages of sequential encounter encoding in reducing computational complexity and improving mining efficiency.

Moving forward, there are several avenues for future research. Expanding the application of sequential encounter encoding to different problem domains and refining the encoding process to accommodate diverse datasets are promising areas for exploration. Additionally, investigating hybrid approaches that combine sequential encounter encoding with other optimization techniques could yield further improvements in mining performance.

Overall, this study contributes to the advancement of itemset mining methodologies and lays the groundwork for future research endeavors aimed at enhancing the efficiency and effectiveness of mining algorithms. By embracing innovative approaches such as sequential encounter encoding, we can unlock new insights from large-scale datasets and drive progress in data mining and analytics.