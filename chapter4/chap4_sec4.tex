\subsection{Làm mịn đường di chuyển của robot với phép nội suy đường khối}
Đường được tạo bởi thuật toán lập kế hoạch đường đi thường là đường thẳng từng đoạn và đôi khi là đường ngoằn ngoèo 90 độ. Do đó, để đi theo những con đường này một cách chính xác, rô bốt di động phải dừng, quay và mất đều đường dẫn đến giảm năng suất. Vì vậy, để khắc phục những nhược điểm này, đường đi cần phải được làm cong đi một chút. Trong bài báo này, phương pháp nội duy spline sẽ được áp dụng để làm mịn đường dẫn. Tập hợp các điểm tham chiếu được tạo bằng cách chia các đường rẽ thành các đoạn bằng nhau tương ứng. Mỗi ba tham chiếu sẽ liên tục được kết nối bằng đa thức nội suy. Bằng cách sử dụng chức năng nội suy 3 bậc, đường ngoằn ngoèo 90 độ trở thành đường cong để robot có thể dễ dàng thực hiện chuyển động dọc theo nó.