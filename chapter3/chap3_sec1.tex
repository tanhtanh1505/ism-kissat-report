In this chapter, we will introduce a method to optimize CNF encodings, which is called Sequential Encounter Encoding (SEE).
Base on the work of Carsten Sinz for problem 'At Most K',
we will adapt the SEE method to solve the problem 'at least k' of itemset mining.


\section{Sequential Encounter Encoding}

In a paper published in 2005, Carsten Sinz\cite{carstensinz} presented a novel approach to encode constraints within the realm of CNF encodings,
which he termed Sequential Encounter Encoding.
This technique introduced a method to count occurrences of boolean variables,
particularly useful for encoding problems where the number of true variables
must not exceed a certain threshold $k$ in the series of variables $x_1, x_2, ..., x_n$.
It is denote as $\le _k(x_1,...,x_n)$, where the goal is to ensure that at most $k$ of the variables $x_1, x_2,..., x_n$ are true.

The approach is to encode a circuit that sequentially counts the number of $x_i$ that are true.
For each $1 \le i \le n$, a register is introduced to store the count of $x_1, x_2,..., x_i$ that are true.
Each register maintains its count in base one, using $k$ bits to count to $k$.
Thus, the encoding introduces new variables $r_{ij}$, $1 \le i \le n$, $1 \le j \le k$, where each $r_{ij}$ represents the $j^{th}$ bit of register $i$.

In section 2, we have encoded the problem of item set mining in terms of constraints \ref{eq:2}, \ref{eq:3} and \ref{eq:4}.
With constraint \ref{eq:4}, we ensure that the number of transactions $q_i$ that contain the itemset $X$ is at least $\lambda$.

Building on Sinz's foundational work, this paper adapts the SEE technique
for an \textbf{'at least k'} problem context,
applying it specifically to CNF encoding in itemset mining.
Instead of ensuring that at most $k$ of the variables are true, the goal is to ensure that at least $k$ of the variables are true.

Within the scope of itemset mining, this means that each transaction $q_i$ is represented by a boolean variable,
true if the transaction contains the itemset $X$ and false otherwise.
The goal is to ensure that at least $\lambda$ (equivalent to $k$)
of these transactions are true within a selected subset of transactions $q_1, q_2,..., q_n$.

To approach this, each transaction is processed one by one, updating a register $r$ which holds the running true transactions up to that point.
If $r_{ij}$ is true, it indicates that by the time the sequence reaches transaction $q_i$, there have been $j$ transactions from $q_1$ to $q_i$ are true.

In adapting Sinz's method for this new application, this paper sheds light on the potential versatility of Sequential Encounter Encoding in addressing a broader range of logical constraints, extending its utility beyond the original 'at most k' problems.