\section{Optimization CNF Encoding using Sequential Encounter Encoding}

We divide the process into two parts. The first part involves constructing the variables $r_{ij}$,
and the second part involves adding the requirement that at least $\lambda$ transactions must be true.
\subsection{Construct register}

First of all, we introduce the following constraint to establish the fundamental relationship between $q$ and $r$:
\begin{equation}
    \label{eq:at_least_one_bit_true_when_q_i_true}
    \bigwedge_{i=1}^{n} \left( q_i \rightarrow r_{i1} \right)
\end{equation}

This constraint ensures that if $q_i$ is true, then the first bit of register $i$ will also be true. In other words, when processing $q_i$ and $q_i$ is true,
we can guarantee that at least one transaction $q$ from $q_1$ to $q_i$ is true.
While there may be more than one transaction that containt the expected itemset, we only ensuring a minimum of one transaction is true.

When $i < \lambda$, if $q_i$ is false, bit $i$ of register $i$ will be false.
Suppose all transactions from $q_1$ to $q_{i-1}$ are true, but $q_i$ is false, then $r_{i,i}$ will be false.
This constraint is expressed as:
\begin{equation}
    \label{eq:bit_i_of_register_i_false_when_q_i_false}
    \bigwedge_{i=1}^{k} \left( \neg q_i \rightarrow \neg r_{ii} \right)
\end{equation}


Nextly, we have a constraint that allows us to disregard cases where the number of processed transactions is less than $\lambda$:
\begin{equation}
    \label{eq:disregard_when_i_less_than_lambda}
    \bigwedge_{i=1}^{\lambda-1} \bigwedge_{j=i+1}^{\lambda} \left( \neg r_{ij} \right)
\end{equation}

This constraint ensures that if $i$ is less than $\lambda$, all bits after $i$ in register $i$ will be false.
In practice, it has been observed that in order to have at least $\lambda$ transactions that are true from $q_1$ to $q_i$,
we need to process at least $\lambda$ transactions.
Therefore, we can disregard cases where the number of processed transactions is less than $\lambda$.

Additionally, we have a constraint that allows us to clone the previous result of the register to the next register:
\begin{equation}
    \label{eq:relationship_between_q_i_and_previous_result}
    \bigwedge_{i=2}^{n} \bigwedge_{j=1}^{\lambda} \left( r_{i-1,j} \rightarrow r_{ij} \right)
\end{equation}

This constraint demonstrates that if there are at least $j$ transactions that are true from $q_1$ to $q_{i-1}$,
it is a fact that there are at least $j$ transactions that are true from $q_1$ to $q_i$, regardless of the value of $q_i$.

Next, as we process the transaction $q_i$ and consider its value, we need to take into account the previous result that does not include the transaction $q_i$.
This leads us to the following constraint:
\begin{equation}
    \label{eq:previous_result_without_q_i_true_and_q_i_true_then_register_true}
    \bigwedge_{i=2}^{n} \bigwedge_{j=2}^{\lambda} \left( q_{i} \wedge r_{i-1,j-1} \rightarrow r_{ij} \right)
\end{equation}

This constraint establishes the relationship between the value of $q_i$ and the previous result
when processing up to $i-1$.
Whether $q_i$ is true or false determines whether the result of register $i$
is true or false.
Additionally, the previous result at bit $i$ influences the result of register $i$ at bit $j$.
In other words, if there are $j-1$ transactions that are true from $q_1$ to $q_{i-1}$,
and $q_i$ is true, then there are $j$ transactions that are true from $q_1$ to $q_i$.

If both $r_{i-1,j}$ and $q_i$ are false, we guarantee that $r_{ij}$ will also be false:
\begin{equation}
    \label{eq:no_change_when_q_i_false}
    \bigwedge_{i=2}^{n} \bigwedge_{j=1}^{\lambda} \left( \neg q_{i} \wedge \neg r_{i-1,j} \rightarrow \neg r_{ij} \right)
\end{equation}

When $q_i$ is false and there are not $j$ transactions that are true from $q_1$ to $q_{i-1}$,
then there will also not be $j$ transactions that are true from $q_1$ to $q_i$.
In other words, the value of $q_i$ being false does not affect the result of register $i$.

To enforce the sequential order of bits in register $i$, make sure not have bit 0 before bit 1 in register $i$, we can use the following constraint:
\begin{equation}
    \label{eq:sequential_order}
    \bigwedge_{i=2}^{n} \bigwedge_{j=2}^{\lambda} \left( \neg r_{i-1,j} \wedge \neg r_{i-1,j-1} \rightarrow \neg r_{ij} \right)
\end{equation}

This constraint ensures that if both bit $j$ and bit $j-1$ of register $i-1$ are false,
then bit $j$ of register $i$ will also be false.
In other words, if transaction $q_{i-1}$ is false and there are not having $j$ true transactions from $q_1$ to $q_{i-1}$,
then when processing the next transaction $q_i$, there will also not be $j$ true transactions from $q_1$ to $q_i$.
The best case scenario is that there can only be $j-1$ true transactions.

\subsection{At least $\lambda$ required}

Firstly, to ensure having extract $\lambda$ transactions contains item set $X$ from $q_1$ to $q_n$, we can impose the following constraint:
\begin{equation}
    \label{eq:ensure_lambda_transactions}
    r_{n-1,\lambda} \vee (q_n \wedge r_{n-1,\lambda-1})
\end{equation}

This ensures that when processing $q_{n-1}$, there are either at least $\lambda$ transactions that are true from $q_1$ to $q_{n-1}$,
or there are $\lambda - 1$ transactions that are true from $q_1$ to $q_{n-1}$ and $q_n$ is true.

Secondly, if the value of $q_n$ is false, to having at least $\lambda$ true transactions,
we need $r_{n-1,\lambda}$ must be true.
This ensures that there are at least $\lambda$ transactions that are true from $q_1$ to $q_{n-1}$.

\begin{equation}
    \label{eq:must_have_true_lambda_transactions_before_q_n_false}
    \neg q_n \rightarrow r_{n-1,\lambda}
\end{equation}

Finally, we have the constraint for all $i > \lambda$
\begin{equation}
    \label{eq:at_least_lambda_transactions}
    \bigwedge_{i=\lambda + 1}^{n} \left( \neg q_i \wedge r_{i,\lambda} \rightarrow r_{i-1,\lambda} \right)
\end{equation}

It ensures sure that if $q_i$ is false but when process to it, there are already having $\lambda$ transactions is true from $q_1$ to $q_i$.
This is only because of the previous transactions from $q_1$ to $q_{i-1}$ has $\lambda$ transactions that are contain item set $X$.

We have completed the process of defining the constraints for solving the
"at least k" problem in itemset mining, replacing equation \ref{eq:4}.
By combining all the constraints together, including equations \ref{eq:2} and \ref{eq:3} from section 2,
we obtain the final CNF encoding for the itemset mining problem.