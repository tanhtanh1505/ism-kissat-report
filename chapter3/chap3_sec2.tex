\section{Old Sequential Counter Encoding}
\subsection{Approach to solve "at most k" problem}
Sinz\cite{carstensinz} introduced an encoding of $\le _k(x_1,...,x_n)$
that works by encoding a circuit that sequentially counts the number of $X_i$ that are true.
For each $1 \le i \le n$, there is a register whose value is constrained to contain the number of
$q_1, ..., q_n$ that are true. Each register maintains its count in base one and hence
uses $k$ bits to count to $k$. Thus the encoding introduces the new variables
$r_{ij}$ with $1 \le i \le n$ and $1 \le j \le k$ where each $r_{ij}$ represents the $i^{th}$ bit of register $j$.
The encoding ensures that at most $k$ of the variables are true.

\begin{equation}
    \label{eq:old_seq_counter_encoding_1}
    \bigwedge_{i=1}^{n-1} \left( \neg q_i \vee r_{i1} \right)
\end{equation}

Formula \ref{eq:old_seq_counter_encoding_1} states that if $q_i$ is true then the first bit of register $i$ must be true.

\begin{equation}
    \label{eq:old_seq_counter_encoding_2}
    \bigwedge_{j=2}^{k} \neg r_{1j}
\end{equation}

Formula \ref{eq:old_seq_counter_encoding_2} ensures that in the first register only the first bit can be true.

\begin{equation}
    \label{eq:old_seq_counter_encoding_3}
    \bigwedge_{i=2}^{n-1} \bigwedge_{j=1}^{k} \left(\neg r_{i-1,j} \vee r_{ij} \right)
\end{equation}

\begin{equation}
    \label{eq:old_seq_counter_encoding_4}
    \bigwedge_{i=2}^{n-1} \bigwedge_{j=2}^{k} \left(\neg q_i \vee \neg r_{i-1,j-1} \vee r_{ij} \right)
\end{equation}

Formula \ref{eq:old_seq_counter_encoding_3} and \ref{eq:old_seq_counter_encoding_4}
together constrain each register $i (1 < i < n)$ to contain the value of
the previous register plus $q_i$.

Finally \ref{eq:old_seq_counter_encoding_5} asserts that there can't be an overflow
on any register as it would indicate that more than $k$ variables are true.

\begin{equation}
    \label{eq:old_seq_counter_encoding_5}
    \bigwedge_{i=1}^{n} \left(\neg q_i \vee r_{i-1,k} \right)
\end{equation}


\subsection{Adapting to solve "at least k" problem}

In the context of itemset mining problems, we often encounter the need to solve the "at least k" constraint.
However, current method sequential counter which introduced by Sinz is implemented only for the "at most k" scenario.
So, we will explain how to transform this to address the corresponding "at least k" requirement.

To address the "at least k true" problem, we can convert it to finding "at most $(n-k)$ false".
To achieve this conversion, we replace every occurrence of $k$ with $(n-k)$ and substitute each variable $q_i$
with its negation. By applying these substitutions, we derive the corresponding formulas as follows:

\begin{itemize}
    \item Replace $k$ with $(n-k)$
    \item Replace $q_i$ with $\neg q_i$
\end{itemize}

So, we can generate corresponding formulas to tackle the "at least k" problem.

Equation \ref{eq:old_seq_counter_encoding_1} becomes:

\begin{equation}
    \label{eq:old_seq_counter_encoding_1_at_least}
    \bigwedge_{i=1}^{n-1} \left( q_i \vee r_{i1} \right)
\end{equation}

Equation \ref{eq:old_seq_counter_encoding_2} becomes:

\begin{equation}
    \label{eq:old_seq_counter_encoding_2_at_least}
    \bigwedge_{j=2}^{n-k} \neg r_{1j}
\end{equation}

Equation \ref{eq:old_seq_counter_encoding_3} becomes:

\begin{equation}
    \label{eq:old_seq_counter_encoding_3_at_least}
    \bigwedge_{i=2}^{n-1} \bigwedge_{j=1}^{n-k} \left( q_i \vee r_{i-1,j} \right)
\end{equation}

Equation \ref{eq:old_seq_counter_encoding_4} becomes:

\begin{equation}
    \label{eq:old_seq_counter_encoding_4_at_least}
    \bigwedge_{i=2}^{n-1} \bigwedge_{j=2}^{n-k} \left( q_i \vee r_{i-1,j-1} \vee r_{ij} \right)
\end{equation}

Equation \ref{eq:old_seq_counter_encoding_5} becomes:

\begin{equation}
    \label{eq:old_seq_counter_encoding_5_at_least}
    \bigwedge_{i=1}^{n} \left( q_i \vee r_{i-1,n-k} \right)
\end{equation}

