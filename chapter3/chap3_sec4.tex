\section{Combination of OSC and NSC}

Combining the Old Sequential Counter (OSC) and New Sequential Counter (NSC) methods results in a comprehensive Sequential Counter (SC) approach tailored for "at least k" constraints in itemset mining.

When using the OSC encoding method, it becomes evident that its drawback is the need for a greater number of constraints and variables as the difference between n and k (n-k) increases.
On the other hand, for smaller values of (n-k), the number of generated constraints and variables decreases.
Similarly, the NSC method has its own strengths and weaknesses depending on the value of k: it performs well with smaller k values but faces challenges as k increases.

The inherent weaknesses of one method coincide with the strengths of the other. By integrating OSC and NSC, the resulting SC method efficiently addresses the "at least k" problem encountered in itemset mining. This approach optimally handles larger-scale itemset mining challenges.

In practice, the selection between OSC and NSC depends on the value range of k (or $\lambda$).
To determine the approximate range, we can consider the number of variables or clauses generated by each method.

For instance, we have the number of clauses generated by OSC and NSC as follows:

\begin{equation*}
    \text{OSC\_clauses} = 2n^2 - 2n - 2nk + 3k
\end{equation*}
\begin{equation*}
    \text{NSC\_clauses} = 4nk - 4k + (k^2+k)/2 + 4
\end{equation*}

Make $\text{OSC\_clauses} = \text{NSC\_clauses}$, we have $\frac{k}{n} \approx 0.324$.
So we can use the following equation to determine which method to use:

\begin{equation}
    \label{eq:sc_osc_nsc}
    SC =
    \begin{cases}
        OSC & \iff \frac{k}{n} \geq 0.324 \\
        NSC & \iff \frac{k}{n} < 0.324
    \end{cases}
\end{equation}
