\section{Technical background}
\subsection{Propositional Expression}
Propositional logic formulas or propositional expressions are constructed from variables
and logical operators AND (conjunction), OR (disjunction), NOT (negation), and
parentheses.

\textbf{Proposition}: Each statement that can be either true or false is called a proposition,
denoted by letters such as P, Q, R, ...

\textbf{Negation}

The negation (NOT) of a proposition P is denoted by $\lnot$ P. The negation
is true when P is false.

Truth table:
\begin{table}[H]
    \centering
    \caption{Truth table of negation (NOT)}
    \label{tab:truth_table_negation}v
    \begin{tabular}{|c|c|}
        \hline
        P & $\lnot$ P \\
        \hline
        T & F         \\
        F & T         \\
        \hline
    \end{tabular}
\end{table}

\textbf{Conjunction}

The conjunction (AND) of two propositions P and Q is denoted by P
$\land$ Q. The conjunction is true only when both P and Q are true.

Truth table:
\begin{table}[H]
    \centering
    \caption{Truth table of conjunction (AND)}
    \label{tab:truth_table_conjunction}
    \begin{tabular}{|c|c|c|}
        \hline
        P & Q & P $\land$ Q \\
        \hline
        T & T & T           \\
        T & F & F           \\
        F & T & F           \\
        F & F & F           \\
        \hline
    \end{tabular}
\end{table}

\textbf{Disjunction}

The disjunction (OR) of two propositions P and Q is denoted by P $\lor$ Q.
The disjunction is true when at least one of P and Q is true.

Truth table:
\begin{table}[H]
    \centering
    \caption{Truth table of disjunction (OR)}
    \label{tab:truth_table_disjunction}
    \begin{tabular}{|c|c|c|}
        \hline
        P & Q & P $\lor$ Q \\
        \hline
        T & T & T          \\
        T & F & T          \\
        F & T & T          \\
        F & F & F          \\
        \hline
    \end{tabular}
\end{table}

\textbf{Implication}

The implication of two propositions P and Q is denoted by P $\rightarrow$ Q.
The implication is false only when P is true and Q is false.

Truth table:
\begin{table}[H]
    \centering
    \caption{Truth table of implication}
    \label{tab:truth_table_implication}
    \begin{tabular}{|c|c|c|}
        \hline
        P & Q & P $\rightarrow$ Q \\
        \hline
        T & T & T                 \\
        T & F & F                 \\
        F & T & T                 \\
        F & F & T                 \\
        \hline
    \end{tabular}
\end{table}

\textbf{Biconditional}

The biconditional of two propositions P and Q is denoted by P $\leftrightarrow$ Q.
The biconditional is true when both P and Q have the same truth value.

Truth table:
\begin{table}[H]
    \centering
    \caption{Truth table of biconditional}
    \label{tab:truth_table_biconditional}
    \begin{tabular}{|c|c|c|}
        \hline
        P & Q & P $\leftrightarrow$ Q \\
        \hline
        T & T & T                     \\
        T & F & F                     \\
        F & T & F                     \\
        F & F & T                     \\
        \hline
    \end{tabular}
\end{table}

\subsection{Conjunction Normal Form (CNF)}

A propositional formula is in conjunctive normal form (CNF) if it is a conjunction of clauses, where each clause is a disjunction of literals. A literal is a propositional variable or its negation.
The standard form of CNF is
\begin{equation*}
    (P_1 \lor P_2 \lor ... \lor P_n)_1 \land ... \land (Q_1 \lor Q_2 \lor ... \lor Q_m)_p \quad n,m,p \geq 1
\end{equation*}

Sample truth table (only for P and Q and R):
\begin{table}[H]
    \centering
    \caption{Truth table of CNF}
    \label{tab:truth_table_cnf}
    \begin{tabular}{|c|c|c|c|}
        \hline
        P & Q & R & (P $\land$ Q) $\land$ R \\
        \hline
        T & T & T & T                       \\
        T & T & F & F                       \\
        T & F & T & F                       \\
        T & F & F & F                       \\
        F & T & T & F                       \\
        F & T & F & F                       \\
        F & F & T & F                       \\
        F & F & F & F                       \\
        \hline
    \end{tabular}
\end{table}

\subsection{Technical background of Itemset Mining}
Firstly, we establish several symbols to represent the itemset mining problem.
These symbols aid in formalizing the problem and defining key concepts.
For instance, we denote:
\begin{itemize}
    \item \textbf{$\Omega$}: a set of all items
    \item \textbf{$I$}: an itemset in $\Omega$, where $I \subseteq \Omega$
    \item \textbf{$T_i$}: a transaction identifier. For $T_i$ = $(i,I)$
    \item \textbf{$D$}: a transaction database, where $D$ contains a set of transactions, $D = \{T_1, T_2, ..., T_n\}$
    \item \textbf{$Supp(I, D)$}: the support of itemset $I$ in database $D$, where $Supp(I, D)$ is the number of transactions in $D$ that contain $I$
\end{itemize}
For example, in table \ref{tab:example_dataset_in_real},
we can present the dataset as a transaction database $D$.

Let $a = apple, b = banana, c = cherry, d = mango$.
Then we have database transactions in binary format as shown in table \ref{tab:example_dataset_convert_to_binary}.

\begin{table}[H]
    \centering
    \caption{Sample dataset of transactions in binary format}
    \label{tab:example_dataset_convert_to_binary}
    \begin{tabular}{|c| c c c c |}
        \hline
        \textbf{Tid} & \textbf{a} & \textbf{b} & \textbf{c} & \textbf{d} \\
        \hline
        1            & 1          & 1          & 1          & 0          \\
        2            & 1          & 0          & 0          & 1          \\
        3            & 1          & 0          & 1          & 0          \\
        4            & 0          & 0          & 1          & 1          \\
        5            & 1          & 0          & 1          & 1          \\
        \hline
    \end{tabular}
\end{table}

\begin{itemize}
    \item $\Omega$ is \{apple, banana, cherry, mango\}
    \item $I$ can be \{apple\}, \{apple, mango\}, \{apple, mango, cherry\}, ...
    \item $D$ = \{(1, \{apple, banana, cherry\}), (2, \{apple, mango\}), (3, \{apple, cherry\}), (4, \{mango, cherry\}), (5, \{apple, mango, cherry\})\}
    \item $T_1$ = (1, \{apple, banana, cherry\}), $T_2$ = (2, \{apple, mango\}), $T_3$ = (3, \{apple, cherry\}),...
    \item $Supp(\{apple, cherry\}, D)$ = 3, $Supp(\{apple\}, D)$ = 4,...
\end{itemize}
Let $\lambda$ be the minimum support threshold,
the frequent itemset mining problem is to find all itemsets $I$ such that $Supp(I, D) \geq minsup$. In general, it can present by:

\begin{center}
    $FIM(D,\lambda)$ = \{I $\subseteq \Omega$ $|$ $Supp(I, D) \geq \lambda$\}
\end{center}

One of the major challenges in itemset mining is the potential exponential growth of the output, even when using condensed representations of patterns.
