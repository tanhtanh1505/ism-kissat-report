In this chapter, we present how to encode the itemset mining problem into a SAT problem. And we will discuss the limitation of the standard method.
\section{Constraint Encoding}
To resolve the itemset mining problem, we using the SAT encoding approach.
In essence, SAT encoding involves the creation of variables and the imposition of constraints to represent the itemset mining problem.
These variables serve to denote the presence or absence of items within a candidate itemset and are subjected to linear inequalities to ensure the itemset's support.

In the context of a transaction database $D$ = {(1,$T_1$),...,(m,$T_m$)} and a minimum support threshold $\lambda$,
each item in the candidate itemset $X$, we denote:
\begin{itemize}
    \item $p_a$: is $true$ if the item $a$ is in the itemset $X$, otherwise $p_a = false$
    \item $q_i$: is $true$ if the transaction $T_i$ contains the itemset $X$, otherwise $q_i = false$
\end{itemize}
Alongside, a set of constraints is imposed on these variables to establish a one-to-one correspondence between the models of the resulting CNF formula and the set of itemsets.

Firstly, to capture all the transactions where the candidate itemset does not appear, we use following constraint:
\begin{equation}
    \label{eq:1}
    \bigwedge_{i=1}^{m} (q_i \leftrightarrow \bigwedge_{a \notin T_i} \neg p_a)
\end{equation}
This constraint guarantees that $q_i$ is true if and only if either all items not in $T_i$ are also not in the itemset $X$, or transaction $T_i$ contains the itemset $X$.

Constraint \ref{eq:1} can be rewritten as follows:
\begin{equation}
    \label{eq:2}
    \bigwedge_{a \in \Omega} \bigwedge_{a \notin T_i} (\neg p_a \vee \neg q_i)
\end{equation}
\begin{equation}
    \label{eq:3}
    \bigwedge_{T_i \in D} ((\bigvee_{a \notin T_i} p_a) \vee q_i)
\end{equation}


Finally, the frequency constraint, can be simply expressed as follows:
\begin{equation}
    \label{eq:4}
    \sum_{i=1}^{m} q_i \geq \lambda
\end{equation}

For example, with a dataset of transactions from a retail store in table \ref{tab:example_dataset_in_real}, we mark:
a = apple, b = banana, c = cherry, d = mango. Then we have database transactions

\begin{table}[H]
    \centering
    \begin{tabular}{|c| c c c c |}
        \hline
        \textbf{Tid} & \textbf{a} & \textbf{b} & \textbf{c} & \textbf{d} \\
        \hline
        1            & 1          & 1          & 1          & 0          \\
        2            & 1          & 0          & 0          & 1          \\
        3            & 1          & 0          & 1          & 0          \\
        4            & 0          & 0          & 1          & 1          \\
        5            & 1          & 0          & 1          & 1          \\
        \hline
    \end{tabular}
    \caption{Example of a dataset of transactions after convert}
    \label{tab:example_dataset_after_convert}
\end{table}

The itemset mining problem will be defined as:
\begin{equation*}
    \begin{aligned}
         & q_1 \leftrightarrow (\neg p_d                ) \\
         & q_2 \leftrightarrow (\neg p_b \wedge \neg p_c) \\
         & q_3 \leftrightarrow (\neg p_b \wedge \neg p_d) \\
         & q_4 \leftrightarrow (\neg p_a \wedge \neg p_d) \\
         & q_5 \leftrightarrow (\neg p_b                ) \\
         & q_1 + q_2 + q_3 + q_4 + q_5 \geq \lambda
    \end{aligned}
\end{equation*}
