This chapter will focuses on introducing the itemset mining tasks and SAT encoding, encompassing their concepts, related terms, and applications.
\section{Frequent Itemset Mining}

Data mining is concerned with either forecasting future trends or deciphering past events. Techniques used for predicting the future, such as neural networks, often function as black-box models because the primary objective is to achieve the highest possible accuracy rather than explainability. On the other hand, various data mining methods aim to uncover patterns in data that are straightforward for humans to interpret.

These methods of pattern discovery can be categorized based on the specific types of patterns they identify, including clusters, itemsets, trends, and outliers. For example, clusters group similar data points together, itemsets identify common associations or groupings in data, trends reveal changes or movements over time, and outliers pinpoint unusual or unexpected data points.

This paper provides a survey that focuses specifically on the discovery of itemsets within databases. Itemset discovery is a popular data mining task, especially when analyzing symbolic data, as it can provide valuable insights into associations and relationships within datasets.

The concept of discovering itemsets in databases was introduced in 1993 by Agrawal and Srikant under the term large itemset mining, which is now known as frequent itemset mining (FIM). The objective of FIM is to identify groups of items (itemsets) that often occur together within customer transactions.

For example, analyzing a customer transaction database may reveal that many customers purchase taco shells along with peppers. Recognizing these associations between items helps to shed light on customer behavior. This knowledge can be invaluable for retail managers, as it enables them to make strategic marketing decisions such as promoting products together or positioning them closer on store shelves. Such strategies can lead to enhanced customer experiences and potentially increased sales.

Frequent itemset mining (FIM) was initially proposed as a method for analyzing customer transaction data, but it has since evolved into a general data mining task that is applicable across various domains. In broader terms, a customer transaction database can be seen as a collection of instances representing objects (transactions), with each object characterized by nominal attribute values (items). As such, FIM can also be understood as the process of identifying attribute values that commonly occur together in a database.

Given that many data types can be represented in the form of transaction databases, FIM finds applications across a diverse range of fields. These include bioinformatics, image classification, network traffic analysis, customer review analysis, activity monitoring, malware detection, and e-learning, among others.

FIM has also been extended in numerous ways to cater to specific requirements and challenges within these domains. For example, extensions of FIM have been developed to discover rare patterns, correlated patterns, patterns in sequences and graphs, and patterns that yield high profit. These adaptations expand the applicability of FIM and demonstrate its versatility and relevance across different areas of data mining.
