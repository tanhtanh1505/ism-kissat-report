This chapter will focuses on introducing the itemset mining tasks and SAT encoding, encompassing their concepts, related terms, and applications.
\section{Itemset Mining Tasks}
\subsection{Overview}
Frequent item sets are a key technique in the realm of data mining,
specifically aimed at uncovering relationships among different items within a dataset.
The essence of association rule mining lies in identifying those item relationships that occur frequently together in the dataset.

In simpler terms, a frequent item set refers to a collection of items that commonly appear together in the dataset.
We measure the frequency of an item set using what's known as the support count.
This count tells us how many times the particular set of items crops up together in transactions or records within the dataset.
In practice, the goal is to find item sets with a minimum support, indicating how frequently they occur in transactions or records within the dataset.

For example, with a dataset of transactions from a retail store
\begin{table}[H]
    \centering
    \begin{tabular}{|c|l|}
        \hline
        \textbf{Tid} & \textbf{Itemsets}     \\
        \hline
        1            & apple, banana, cherry \\
        2            & apple, mango          \\
        3            & apple, cherry         \\
        4            & mango, cherry         \\
        5            & apple, mango, cherry  \\
        \hline
    \end{tabular}
    \caption{Example of a dataset of transactions}
    \label{tab:example_dataset_in_real}
\end{table}

With minimum support is 3, we need to find all itemsets appearing in at least 3 transactions and return the following result:

\begin{itemize}
    \item Itemset 1: \{apple, cherry\} in transactions [1, 3, 5]
    \item Itemset 2: \{apple\} in transactions [1, 2, 3, 5]

\end{itemize}

\subsection{Technical background}
Firstly, we establish several symbols to represent the itemset mining problem.
These symbols aid in formalizing the problem and defining key concepts.
For instance, we denote:
\begin{itemize}
    \item \textbf{$\Omega$}: a set of all items
    \item \textbf{$I$}: an itemset in $\Omega$, where $I \subseteq \Omega$
    \item \textbf{$T_i$}: a transaction identifier. For $T_i$ = $(i,I)$
    \item \textbf{$D$}: a transaction database, where $D$ contains a set of transactions, $D = \{T_1, T_2, ..., T_n\}$
    \item \textbf{$Supp(I, D)$}: the support of itemset $I$ in database $D$, where $Supp(I, D)$ is the number of transactions in $D$ that contain $I$
\end{itemize}
For example, in table \ref{tab:example_dataset_in_real}, we have:
\begin{itemize}
    \item $\Omega$ is \{apple, banana, cherry, mango\}
    \item $I$ can be \{apple\}, \{apple, mango\}, \{apple, mango, cherry\}, ...
    \item $D$ = \{(1, \{apple, banana, cherry\}), (2, \{apple, mango\}), (3, \{apple, cherry\}), (4, \{mango, cherry\}), (5, \{apple, mango, cherry\})\}
    \item $T_1$ = (1, \{apple, banana, cherry\}), $T_2$ = (2, \{apple, mango\}), $T_3$ = (3, \{apple, cherry\}),...
    \item $Supp(\{apple, cherry\}, D)$ = 3, $Supp(\{apple\}, D)$ = 4,...
\end{itemize}
Let $\lambda$ be the minimum support threshold,
the frequent itemset mining problem is to find all itemsets $I$ such that $Supp(I, D) \geq minsup$. In general, it can present by:

\begin{center}
    $FIM(D,\lambda)$ = \{I $\subseteq \Omega$ $|$ $Supp(I, D) \geq \lambda$\}
\end{center}
